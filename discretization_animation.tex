\begin{figure}
  \centering
  \begin{tikzpicture}
    \begin{axis}
      \addplot[domain=-5:5,samples=100, very thick] function {1/(sqrt(2*3.1416))/1.5*exp(-(x-2)**2/2)+0.5/1.5/(sqrt(2*3.1416))*exp(-(x+1)**2/2)};
      % \pgfmathparse{exp(-x^2)} \draw[dashed] node[right] {x,\pgfmathresult}
      \addplot+[only marks]  coordinates {
        (2.537667, 0)
        (3.833885, 0)
        (-0.258847, 0)
        (2.862173, 0)
        (2.318765, 0)
        (0.692312, 0)
        (1.566408, 0)
        (2.342624, 0)
        (5.578397, 0)
        (4.769437, 0)
        (0.650113, 0)
        (5.034923, 0)
        (2.725404, 0)
        (1.936945, 0)
        (2.714743, 0)
        (1.795034, 0)
        (1.875856, 0)
        (3.489698, 0)
        (3.409034, 0)
        (3.417192, 0)
        (2.671497, 0)
        (0.792513, 0)
        (2.717239, 0)
        (3.630235, 0)
        (2.488894, 0)
        (3.034693, 0)
        (2.726885, 0)
        (1.696559, 0)
        (2.293871, 0)
        (1.212717, 0)
        (-0.111604, 0)
        (-2.147070, 0)
        (-2.068870, 0)
        (-1.809499, 0)
        (-3.944284, 0)
        (0.438380, 0)
        (-0.674809, 0)
        (-1.754928, 0)
        (0.370299, 0)
        (-2.711516, 0)
        (-1.102242, 0)
        (-1.241447, 0)
        (-0.680793, 0)
        (-0.687141, 0)
        (-1.864880, 0)
      };
    \end{axis}
  \end{tikzpicture}
  \begin{tikzpicture}
    \begin{axis}
      \addplot+[only marks] coordinates {
        (-3.9443,0)
        (-1.2414,0)
        (0.9890,0)
        (2.7147,0)
        (3.8339,0)
      };
      \addplot[fill=blue!20] coordinates {
        (-5.2957,0.0242)
        (-2.5929,0.0242)
      }
      |- (axis cs:-5.2957,0) -- cycle;
      \addplot[fill=blue!60] coordinates {
        (-2.5929,0.1449)
        (-0.0144,0.1449)
      }
      |- (axis cs:-2.5929,0) -- cycle;
      \addplot[fill=blue!20] coordinates {
        (-0.0144,0.1329)
        (1.9637,0.1329)
      }
      |- (axis cs:-0.0144,0) -- cycle;
      \addplot[fill=blue!60] coordinates {
        (1.9637,0.1449)
        (3.2743,0.1449)
      }
      |- (axis cs:1.9637,0) -- cycle;
      \addplot[fill=blue!20] coordinates {
        (3.2743,0.0966)
        (4.3935,0.0966)
      }
      |- (axis cs:3.2743,0) -- cycle;
      \addplot[no marks, samples=100, color=black, very thick] function {1/(sqrt(2*3.1416))/1.5*exp(-(x-2)**2/2)+0.5/1.5/(sqrt(2*3.1416))*exp(-(x+1)**2/2)};
    \end{axis}
  \end{tikzpicture}
  \begin{tikzpicture}
    \begin{axis}
      \addplot[no marks, samples=100, color=black, fill=blue!60, very thick] function {1/(sqrt(2*3.1416))/1.5*exp(-(x-2)**2/2)+0.5/1.5/(sqrt(2*3.1416))*exp(-(x+1)**2/2)};
      \addplot[no marks, samples=100, color=black, fill=red!40, very thick] function
      {1/(sqrt(2*3.1416))/1.5*exp(-(x-2)**2/2)};
    \end{axis}
  \end{tikzpicture}
  \caption{Top left: Original probability distribution. The dots represent random samples from this distribution. Top right: K-means approximation with five clusters for the 45 data points on the left plot. Note that the approximation completely fails to imitate the structure of the distribution. Bottom: The probability distribution is a mixture of Gaussians.}
  \label{fig:rev-disc-xiorig}
\end{figure}

%%% Local Variables:
%%% mode: latex
%%% TeX-master: "da"
%%% End:
