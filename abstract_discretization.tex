\pgfdeclarelayer{background}
\pgfsetlayers{background,main}
\begin{figure}
  \centering
    \begin{tikzpicture}[node distance=5cm, auto]
      \node [block] (xiorig) {$\hat{\bs\xi}$};
      \node [block, below of=xiorig] (xi) {$\bs\xi$};
      \node [block, text width=10em, right of=xiorig] (infsolver){Infinite Dim. Solver};
      \node [block, right of=infsolver] (xorig){$\hat{\bs x}$};
      \node [block, text width=10em, below of=infsolver] (nlp) {(N)LP Solver};
      \node [block, below of=xorig] (x) {$\bs x$};

      \path [line] (xiorig) -- node (xierror) {$D(\bs\xi,\hat{\bs\xi})$} (xi);
      \path [line] (x) -- node (xerror) {$D(\bs x,\hat{\bs x})$} (xorig);
      \path [line,dashed] (xierror) -- node {$L\cdot D(\bs\xi,\hat{\bs\xi})\geq D(\bs x,\hat{\bs x})$} (xerror);
      \path [line] (xiorig) -- (infsolver);
      \path [line] (xi) -- (nlp);
      \path [line] (nlp) -- (x);
      \path [line] (infsolver) -- (xorig);

      \node [node distance=1.4cm,below of=xi] (focus){Focus of this paper};
      \begin{pgfonlayer}{background}
        \node (selection) [draw, rounded corners,dotted, fill=orange!20, fit=(xiorig) (xi) (xierror) (focus)] {};
      \end{pgfonlayer}
    \end{tikzpicture}
  \caption{Steps in the discretization of a stochastic program.
    Since the infinite dimensional solver is just a theoretical entity, the path below must be used.
    A discretization error $D(\bs\xi,\hat{\bs\xi})$ is introduced when converting $\hat{\bs\xi}$ to $\bs\xi$.
    This error is propagated through the NLP.
    For linear stochastic programs, it has been shown %Heitsch2010
    that there exists a problem specific constant $L$ such that the error $D(\bs x, \hat{\bs x})$ between the discretized and the original solution is bounded by $D(\bs\xi, \hat{\bs\xi})$.}
  \label{fig:abstract-discretization}
\end{figure}

%%% Local Variables:
%%% mode: latex
%%% TeX-master: "da"
%%% End:
