\section{Introduction}
Many applications in engineering and economics require decisions to be made under uncertainty.
Examples are production planning and supply chain deign decisions under uncertain demands and prices, but also robust inventory planning for the case of plant failures.

Stochastic programming problems are optimization problems under uncertainty, which require a recourse decision after the uncertainty variable has taken on its value.
Problems in which a sequence of such decisions has to be made are called multi-stage stochastic programs.

In this work, two problems in the formulation of multi-stage stochastic programs will be addressed.
in real-world applications, the uncertainty often is introduced as a continuous random variable.
Since in the optimization the recourse problem has to be solved for every possible outcome of the random variable, this leads to infinite dimensional problems.
Even if the underlying random variable is discrete, for multi-stage problems the number of variables grows exponentially with the number of stages.
Methods have to be developed that reduce the complexity of the problem while sacrificing as little accuracy as possible.
An important aspect of this reduction is the preservation of the way that the information about the future over time.
This leads to tree-shaped data structures through which this filtration is encoded.
These properties, that are special to multi-stage problems, makes even measuring the quality of a reduction a difficult enterprise.

This thesis is organized as follows.
First, the theoretical groundwork for the scenario generation is laid out.
The problem of measuring the quality of an approximation is formalized through a metric.
The optimal scenario tree generation will prove to be representable as an MILP/NLP.
In a following section, state-of-the-art MILP and NLP solvers are applied to the problem.
Due to structural reasons, this approach will prove insufficient.
In section \ref{sec:two-theorems}, the derivation of the new method will begin with two theorems which highlight the special structure of the used metric.
The structure of the metric exposed through these theorems will be exploited by the new heuristic algorithm derived in section \ref{sec:expect-max-algos}.
This algorithm is based on the well-known K-Means algorithm.
Promising results will be presented, that show the scalability of the algorithm, and that the quality of the solution is near optimal.
Building on the derivation of the new algorithm, the application of Expectation-Maximization algorithms for the tree generation will be discussed.
%%% Local Variables:
%%% mode: latex
%%% TeX-master: "da"
%%% End:
