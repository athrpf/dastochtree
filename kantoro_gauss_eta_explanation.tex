\newcommand{\fdst}[7]{%mu, sigma, outer bounds, start of marker, scale, y
  % shade the critical region tail
  \draw[fill,orange]  (#4+#1,#6) node[below]{#7}-- plot[domain=(#4+#1):(#4+#1+0.2*#2),samples=50]
  function {#6+#5*1/sqrt(2*3.1416*#2)*exp(-(x-#1)*(x-#1)/2/#2)}
  -- (#4+#1+0.2*#2,#6) -- cycle;

  % draw the F distribution curve
    \draw[color=blue!50!black,thick]
        plot[smooth,domain=(-#3+#1):(#3+#1),samples=100]
        function {#6+#5*1/sqrt(2*3.1416*#2)*exp(-(x-#1)*(x-#1)/2/#2)};


    % draw the F axis
        %\draw[->] (-#3,0) -- (#3,0) node[right] {$F$};
    % label the critical region boundary
        %    \draw (0,0) -- (#1,-0.02) node[below] {$#1$};
    % label 0
%    \draw (0,0) -- (0,-0.02) node[below] {$0$};

    % draw the y axis
 %   \draw[very thin,->] (0,0) -- (0,#5/2);
}

\begin{figure}
  \centering
  \begin{tikzpicture}
    \draw[fill,orange]  (1+0,5) node[below]{$x$}-- plot[domain=(1+0):(1+0+0.2*2),samples=50]
  function {5+4*1/sqrt(2*3.1416*2)*exp(-(x-0)*(x-0)/2/2)}node(topnische)[above, left]{}
  -- (1+0+0.2*2,5)  -- cycle;

  % draw the F distribution curve
  \draw[color=blue!50!black,thick]
  plot[smooth,domain=(-5+0):(5+0),samples=100]
  function {5+4*1/sqrt(2*3.1416*2)*exp(-(x-0)*(x-0)/2/2)} node[above] {$\mathbb{P}(x)$};



%    \fdst{0}{2}{5}{1}{3}{5}{$x$};
    % draw upper x- axis
  \draw[->] (-5,5) -- (8.2,5);

    \draw[fill,orange]  (0.707+5,0) node[below]{$ y=\sqrt{\frac{\sigma_n}{\sigma_k}}(x-\mu_k)+\xi_n$} --  plot[domain=(0.707+5):(0.707+5+0.2*0.5),samples=50]
  function {0+5*1/sqrt(2*3.1416*0.5)*exp(-(x-5)*(x-5)/2/0.5)} node(bottomnische)[below]{}
  -- (0.707+5+0.2*0.5,0)  -- cycle;

  % draw the F distribution curve
    \draw[color=blue!50!black,thick]
        plot[smooth,domain=(-3+5):(3+5),samples=100]
        function {0+5*1/sqrt(2*3.1416*0.5)*exp(-(x-5)*(x-5)/2/0.5)} node[above]{$\mathbb{Q}(y)$};

        % \fdst{5}{0.5}{3}{0.707}{5}{0}{{$ y=\sqrt{\frac{\sigma_2}{\sigma_1}}(x-\mu_1)+\mu_2$}};
        \draw[->] (-5,0) -- (8.2,0);

        \path[->,thick, color=blue!50] (topnische) edge[bend right] node[right]{$\eta(x,y)$}(bottomnische) ;

  \end{tikzpicture}
  \caption{The measure $\eta$ for two Gaussian distributions}
  \label{fig:kantoro_gauss_explain}
\end{figure}

%%% Local Variables:
%%% mode: latex
%%% TeX-master: "da"
%%% End:
