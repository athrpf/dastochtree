\section{Conclusion}
The discretization of stochastic processes is an important step in the solution of stochastic programs. In this thesis, three advances for the computation of this discretization are presented.

First, a rigorous MILP/NLP model that characterizes the optimal tree solution was developed. The model is the combination of a selection problem whose objective is again given by the linear transportation problem of the Kantorovich distance. State-of-the-Art mixed integer solvers were used to solve this model. While it is too large to be solved for any practical problem, it can serve as a benchmark when developing new algorithms.

Second, using properties of the MILP characterization, the K-Means-for-trees algorithm has been derived. Using the benchmark algorithms developed before, the quality of these heuristic solutions has been found to match that of the benchmark within a close margin. The algorithm was translated into the more general framework of the Expectation Maximization algorithms. As an example, the tree generation was derived for a Gaussian Mixture model. It has been shown that the Gaussian Mixture model tree corresponds to a function space approximation of the stochastic process. Since the space in which the stochastic process is approximated acts as an implicit bias on the structure of the underlying probability distribution, in many cases where domain knowledge about the distribution is available, a superior approximation can be found.

Finally, it has been shown how an approximation of the discretization error in the objective of the stochastic program can be computed using a tree computed with the Expectation Maximization algorithm. This approximation connects the error in the approximation of the tree with the error in the solution of the stochastic program and assists in evaluating whether a discretization is sufficiently dense.


\paragraph{Acknowledgements}
I would like to acknowledge several software contributions that were helpful for implementing the algorithms of this paper: Gurobi, made available by Gurobi Inc. through Gurobi Optimization. The Gurobi Matlab interface, written by Wotao Yin. The CLP Matlab interface written by Johann L\"{o}fberg. The Kmeans++ algorithm implemented for Matlab by Laurent Sorber. The Kmedoids algorithm implemented in Matlab by Benjamin Sapp.
%%% Local Variables:
%%% mode: latex
%%% TeX-master: "da"
%%% End:
